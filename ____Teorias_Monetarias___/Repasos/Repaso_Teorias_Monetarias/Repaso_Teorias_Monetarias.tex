\documentclass[openany]{book}

\usepackage{generalsnips}
\usepackage{calculussnips}
\usepackage[margin = 1in]{geometry}
\usepackage{pdfpages}
\usepackage[spanish]{babel}
\usepackage{amsmath}
\usepackage{amsthm}
\usepackage[utf8]{inputenc}
\usepackage{titlesec}
\usepackage{xpatch}
\usepackage{fancyhdr}
\usepackage{tikz}
\usepackage{hyperref}
\pagestyle{empty}

\title{Summary on Teorías monetarias}
\begin{document}
%%%%%%%%%%%%%%%%%%%%%%%%%%%%%%%%%%%%%%%%%%%%%%%%%%%%%%%%%%%%%%%%%%%%%%%%%%%%%%%%%%%%%%%%%%%%%%%%%%%%%%%%%%%%%%%%%%%%%%%%%%%%%%%%%%%%%%%%%%%%%%


\chapter{Presentation 3}
\section{Unemployment rate}
\begin{itemize}
    \item Employed: Currently working 
    \item Unemployed: out of work and actively looking for a job.
    \item Labor force = Employed + Unemployed 
    \item Unemplyment rate: 
        \[
          \text{ Unemployment rate } = \frac{\text{ Unemployed people }}{\text{ Total labor force }} \times 100 
        \]
\end{itemize}

\subsection{Hidden unemployment}
Mislabeled people: 
\begin{itemize}
    \item Part-time: temporary.
    \item Underemployed: economist working at mc donald's.
    \item Discouraged workers: those who have stopped looking for employment due to the lack of suitable positions available.
    \item *Extra Transition: person who quits their job to go to another one. 
\end{itemize}

\subsection{Labor force participation}
Labor force participation rate: in proportion to all adults in a country how many people are in the labor force (can work). 
\[
  \text{ Labor force participation rate } = \frac{\text{ Total labor force }}{\text{ Total adult population }} \times 100
\]

\subsection{Patterns of unemployment}
Unemployment moves up and down as the economy moves in and out of recessions and business cycles.

\section{Unemployment facts}
\begin{itemize}
    \item Unemployment in a gender comparison are relatively equal, and they used to be lower for men. 
    \item Unemployment tends to be higher in ages 16-19.
\end{itemize}


%----------------------------------------------------------------------------------------
\section{Cyclical unemployment}
\begin{itemize}
    \item Cyclical unemployment: closely related to the business cycle, higher unemployment during a recession is cyclical unemployment. 
\end{itemize}


%----------------------------------------------------------------------------------------
\section{Unemployment and equilibrium in the labor market}
\begin{itemize}
    \item Labor market is the same as any other market with the subtle difference of the y-axis having wage rate and the x-axis having Quantity of labor. 
\end{itemize}

\subsection{Sticky wages}
\begin{itemize}
    \item The minimum wage creates sticky wages above the equilibrium. 
\end{itemize}


%----------------------------------------------------------------------------------------
\section{Changes in unemployment on the long run}
Natural rate of unemployment: 
\begin{itemize}
    \item Frictional unemployment: unemployment ``between jobs''.
    \item Structural unemployment: individuals lack skills valued by employers thus they are unemployed. 
\end{itemize}
Full unemployment: when the unemployment rate is equal to the natural unemployment rate. 

\subsection{Productivity shifts and the natural rate of unemployment}
\begin{itemize}
    \item At some point the wage rate (based on productivity) will be grater than the demand of labor (grater than the optimal point).
    \item This increase in wage and demand will eventually create unemployment. 
\end{itemize}

\chapter{Presentation 4}
\section{Tracking inflation}
\begin{itemize}
    \item Inflation: general rise in the level of prices in an entire economy.  
    \item Basket of goods and services: hypothetical group of items, with specified quantities of each one meant to represent a ``typical'' set of consumer purchases.
        \begin{itemize}
            \item Used to calculate price levels. 
        \end{itemize}
\end{itemize}

%----------------------------------------------------------------------------------------
\section{Index Numbers}
\begin{itemize}
    \item Index number: a unit-free number derived from the price level over a number of years, which makes computing inflation rates easier, since the index number has values around 100.
        \begin{itemize}
            \item It doesn't have any unit.
        \end{itemize}
    \item Base year: arbitrary year whose value as an index number economists define as 100.
\end{itemize}
Inflation: 
\[
  \text{ Percentage change } = \frac{\p{\text{ Level in new year } - \text{ Level in prior year }} }{\text{ Level in prior year }} \times 100
\]

%----------------------------------------------------------------------------------------
\section{Measure changes in the cost of living}
\begin{itemize}
    \item Consumer price index (CPI): to measure inflation government statisticians calculate based on the price level from a fixed basket of goods and services that represent an average consumer's purchases. 
    \item Substitution bias: 
    \item Quality / new goods bias:  
\end{itemize}


%----------------------------------------------------------------------------------------
\section{The confusion over inflation}
\begin{itemize}
    \item The problem with inflation is that it doesn't sync in real time to measurements, this causes economic problems:
    \begin{itemize}
            \item Unintended redistributions of purchasing power.
            \item Blurred price signals. 
            \item Difficulties in long-term planning.  
        \end{itemize}    
    \item There is a time lag in prices, wages and interest rates. 
\end{itemize}


%----------------------------------------------------------------------------------------
\section{Real interest rate}
Real interest paid (Fischer equation): 
\[
  i_{rt} = r - \pi  
\]
\begin{itemize}
    \item $i_{rt}$ Long term nominal interest rate 
    \item $r$: Nominal interest rate 
    \item $\pi$: actual or expected rate of inflation or deflation
\end{itemize}












%%%%%%%%%%%%%%%%%%%%%%%%%%%%%%%%%%%%%%%%%%%%%%%%%%%%%%%%%%%%%%%%%%%%%%%%%%%%%%%%%%%%%%%%%%%%%%%%%%%%%%%%%%%%%%%%%%%%%%%%%%%%%%%%%%%%%%%%%%%%%%
\end{document}

