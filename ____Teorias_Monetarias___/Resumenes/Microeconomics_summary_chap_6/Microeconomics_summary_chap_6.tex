\documentclass{article}

\usepackage{generalsnips}
\usepackage{calculussnips}
\usepackage[margin = 1in]{geometry}
\usepackage{pdfpages}
\usepackage[spanish]{babel}
\usepackage{amsmath}
\usepackage{amsthm}
\usepackage[utf8]{inputenc}
\usepackage{titlesec}
\usepackage{xpatch}
\usepackage{fancyhdr}
\usepackage{tikz}
\usepackage{hyperref}
\title{Microeconomics summary chap 6}
\date{2020 May 30, 06:12PM}
\author{David Gabriel Corzo Mcmath}

\begin{document}
\maketitle
%%%%%%%%%%%%%%%%%%%%%%%%%%%%%%%%%%%%%%%%%%%%%%%%%%%%%%%%%%%%%%%%%%%%%%%%%%%%%%%%%%%%%%%%%%%%%%%%%%%%%%%%%%%%%%%%%%%%%%%%%%%%%%%%%%%%%%%%%%%%%%

\section{How is the economy doing? How can we tell?}
\begin{itemize}
    \item During the great depression, \textbf{Simon Kuznets} won the novel price for comming up with a way of measuring how bad the situation was, his invention was GDP.
\end{itemize}


%----------------------------------------------------------------------------------------
\section{Introduction to the macroeconomics perspective}
\begin{itemize}
    \item Macroeconomics involves adding up the economic activity of all households and all businesses in all markets to obtain the overall demand and supply in the economy.
    \item What seems sensible from a microeconomic point of view can have unexpected or counterproductive results at the macroeconomic level.
    \item Three different perspectives of macroeconomics as a subject of study: 
        \begin{enumerate}
            \item \textbf{Goals} (what are the goals): Economic growth, low unemployment, low inflation.
            \item \textbf{Framework} (in order to analyze the economy): Aggregate demand / aggregate supply, Keynesian model, Neoclassical model. 
            \item \textbf{Policy Tools} (policy tools for governments to use): Monetary policy, Fiscal policy.  
        \end{enumerate}
\end{itemize}

\subsection{Goals: economic growth}
\begin{itemize}
    \item Determines the prevailing standard of living in a country.
    \item Economists measure growth by the percentage change in real (inflation-adjusted) gross domestic product.
    \item Growth rate of more than 3\% is considered good. 
\end{itemize}

Economic growth: 
\begin{itemize}
    \item Frameworks are usually theories and models. 
    \item In microeconomics we had supply and demand, in macroeconomics we have aggregate supply and aggregate demand.
    \item AS and AD have two perspectives: the Keynesian and the neoclassical.  
\end{itemize}
Unemployment: 
\begin{itemize}
    \item Measured by the unemployment rate. 
    \item Unemployment is unlikely to be zero.
    \item Economists consider a measured unemployment rate of 5\% or less low (good).
\end{itemize}
Inflation: 
\begin{itemize}
    \item Measured by the consumer price index.
    \item If prices are rising much faster than the wages workers receive for their labor, there will be widespread unhappiness as their standard of living declines
    \item Low inflation—an inflation rate of 1–2\%—is a major goal. 
\end{itemize}

\subsection{Frameworks}
\begin{itemize}
    \item The principal tool are theories that interpret aggregate supply and aggregate demand, two interpretations in this book are the Neoclassical and the Keynesian.
\end{itemize}

\subsection{Policy tools}
\begin{itemize}
    \item Monetary policy: managing the money supply and interest rates.
    \item Fiscal policy: changes in government spending/purchases and taxes.
\end{itemize}

%----------------------------------------------------------------------------------------
\section{Measuring the size of the economy: gross domestic product}
\begin{itemize}
    \item GDP is the way to measure the size of the economy. 
    \item GDP: the value ot all final goods and services produced within a country in a given year. 
    \item GDP can be measured by:  total dollar value of what consumers purchase in the economy; or;  total dollar value of what is the country produces. And adding all the income from all the businesses.
\end{itemize}


%----------------------------------------------------------------------------------------
\subsubsection{GDP measured by components of demand}
Who buys all of the production is divided in to four groups: 
\begin{enumerate}
    \item Consumer spending (consumption): about 2/3 of GDP, it doesn's fluctuate a lot, it changes gradually over time. 
    \item Business spending (investment): about 15\% to 18\% of GDP, it fluctuates more noticeably than consumption, this is because the unpredictable changes in technology and/or consumer confidence.
    \item Government spending on goods and services: slightly under 20\%, it includes spending on all levels (federal, state and local), government also gives benefits for social security and other things, this is not accounted for inside GDP because it doesn't produce and good or service, what's included in GDP are things like a new public school construction, a new fighter jet for the Air Force.
    \item Spending on net exports: net exports are all exports minus the imports.
        \begin{itemize}
            \item We call the difference or gap between the imports and the exports the \textbf{trade balance}.
            \item if \textbf{(exports $>$ imports)} then a \textbf{trade surplus} exists.
            \item  if \textbf{(exports $<$ imports)} then a \textbf{trade deficit} exists.
            \item If exports and imports are equal, foreign trade has no effect on total GDP. (Since exports and imports would be zero). How ever foreign trade can have an influence on the country even if the country is balanced. 
        \end{itemize}
\end{enumerate}
Based on the four components of demand: 
\[
  \text{ GDP } = \text{ Consumption } + \text{ Investment } + \text{ Government Spending } + \text{ Trade balance }
\]
\[
  \text{ GDP } = C + I + G + (X - M)
\]

\subsubsection{What does the word ``investment'' mean?}
\begin{itemize}
    \item It refers to purchasing new capital goods, new commercial real estate, for example: buildings, factories, stores, equipment, residential housing, construction, and inventories.
    \item Inventories are included even if they haven't yet been sold. 
    \item Investment does \textbf{not} mean: purchasing stocks and bonds or trading financial assets.  
\end{itemize}

% \subsubsection{How do statitians measure GDP?}
% \begin{itemize}
%     \item 
% \end{itemize}

%----------------------------------------------------------------------------------------
\subsection{GDP Measured by what is produced}
What countries produce are divided into five categories: 
\begin{itemize}
    \item Durable goods: this category has been increasing.
    \item Nondurable goods: this category has been dropping.
    \item Services: this category has been increasing. 
    \item Structures: span everything from buildings, shoping malls and factories.
    \item Change in inventories: this is the smallest category, it includes all the inventory which has not yet been sold. Inventories rise if businesses are bad. 
\end{itemize}
GDP measured according to what is produced is exactly equal the same as the GDP measured by looking at the five components of demand. This is because every transaction must have a \textbf{buyer} and a \textbf{seller}.

\subsubsection{Another way to measure GDP: National income approach}
\begin{itemize}
    \item Add all the income produced in a yeat provides a second way of measuring GDP.
    \item This is why the terms GDP and national income are sometimes used interchangeably. 
\end{itemize}


%----------------------------------------------------------------------------------------
\subsection{The problem of double counting}
\begin{itemize}
    \item To avoid this problem, which would overstate the size of the economy considerably, government statisticians count just the value of final goods and services in the chain of production that are sold for consumption, investment, government,and trade purposes.
    \item Statisticians exclude intermediate goods, which are goods that go into producingothergoods,from GDP calculations. 
\end{itemize}


%----------------------------------------------------------------------------------------
\subsection{Other ways to measure the economy}
Gross National Product (GNP): 
\begin{itemize}
    \item GNP: based more on what a country's citizens and firms produce, wherever they are located.
    \item GDP vs. GNP: GDP is strictly what a country produces inside the borders, GNP is what is produced inside and outside of the borders.  
\end{itemize}
Net National Product (NNP): 
\begin{itemize}
    \item NNP: is the value of the GNP minus the depreciation. 
\end{itemize}


%----------------------------------------------------------------------------------------
\section{Adjusting Nominal values to Real Values}
\begin{itemize}
    \item Nominal value: measure the statistic in terms of actual prices that exist at the time.
    \item Real value: the same statistic after it has been adjusted for inflation. 
    \item GDP Deflator:  is a price index measuring the average prices of all goods and services included in the economy. 
\end{itemize}
\[
  \text{ Real GDP } = \frac{\text{ Nominal GDP }}{\text{ Price Index }} 
\]
Round to two decimal places or multiply times 100.
\[
  \text{ Real GDP } = \frac{\text{ Nominal GDP }}{\p{\cfrac{\text{ Price Index }}{100}}} 
\]

\subsubsection{Find the real GDP growth rate}
Computing change or real GDP growth between 1960 to 2010:  
\[
  \frac{\text{ 2010 real GDP } - \text{ 1960 real GDP}}{\text{ 1960 real GDP }} \times 100 = \text{ \% change } 
\]

Nominal: 
\[
  \text{ Nominal } = \text{ Price } \times \text{ Quantity } 
\]
\[
  \text{ \% change in Nominal } = \text{ \% change in price } + \text{ \% change in quantity }
\]
or 
\[
  \text{ \% change in quantity } = \text{ \% change in nominal } - \text{ \% change in Price }
\]

\section{Tracking Real GDP over time}
\begin{itemize}
    \item GDP is reported anually, however a GDP number is reported every quarter, as a quarter GDP number is compiled it's multiplied by four to report it as annual.
    \item We call a significant decline in real GDP a recession. A lenghty and deep recession a depression. 
    \item Peak: the highest point in the economy before the recession begins. 
    \item Trough: lowest point of the recession before recovery.
    \item Business cycle: the economy's movement from peak to trough and trough to peak.
\end{itemize}

\section{Comparing GDP among countries}
\begin{itemize}
    \item Problems: Different populations and different currencies.
    \item Exchange rate: value of a currency in terms of another. There are two types, market exchange rate and purchasing power parity (PPP).
\end{itemize}
\[
    \text{ Brasil's GDP in \$ U.S } = \frac{\text{ Brazil's GDP in reals }}{\text{ Exchange rate (reals / \$U.S) }} 
\]



%----------------------------------------------------------------------------------------
\section{GDP per capita}
GDP per capita: 
\[
  \text{ GDP per capita } = \frac{\text{ GDP }}{\text{ Population }} 
\]

%----------------------------------------------------------------------------------------
\section{How well GDP Measures the well-being of society}
\begin{itemize}
    \item Standard of living: includes all elements that affect people’s well-being, whether they are bought and sold in the market or not. 
\end{itemize}
GDP does not measure: 
\begin{itemize}
    \item Leisure time 
    \item Levels of environmental cleanliness, health and learning
    \item Life expectancy, infant mortality, literacy rates 
    \item Production that isn't exchanged in the market, autoproduction
    \item Inequality in a society 
    \item Variety available 
    \item Technology and products available 
\end{itemize}
It is possible for GDP to rise and standard of living decrease.
%----------------------------------------------------------------------------------------
\section{Does a rise in GDP overstate or understate the rise in the standard of living?}
\begin{itemize}
    \item GDP doesn't measure a lot of things, thus it isn't a good indicator for measuring standards of living, it is helpful for measuring production, measuring if we are materially better off on terms of jobs and incoms. 
    \item No single number can capture all the elements of a term as broad as “standard of living.” Nonetheless, GDP per capita is a reasonable, rough-and-ready measure of the standard of living.
\end{itemize}



%%%%%%%%%%%%%%%%%%%%%%%%%%%%%%%%%%%%%%%%%%%%%%%%%%%%%%%%%%%%%%%%%%%%%%%%%%%%%%%%%%%%%%%%%%%%%%%%%%%%%%%%%%%%%%%%%%%%%%%%%%%%%%%%%%%%%%%%%%%%%%
\end{document}

