\documentclass{article}

\usepackage{generalsnips}
\usepackage{calculussnips}
\usepackage[margin = 1in]{geometry}
\usepackage{pdfpages}
\usepackage[spanish]{babel}
\usepackage{amsmath}
\usepackage{amsthm}
\usepackage[utf8]{inputenc}
\usepackage{titlesec}
\usepackage{xpatch}
\usepackage{fancyhdr}
\usepackage{tikz}
\usepackage{hyperref}
\title{Resumen videos Clynton}
\date{2020 June 21, 04:20PM}
\author{David Gabriel Corzo Mcmath}

\begin{document}
\maketitle
%%%%%%%%%%%%%%%%%%%%%%%%%%%%%%%%%%%%%%%%%%%%%%%%%%%%%%%%%%%%%%%%%%%%%%%%%%%%%%%%%%%%%%%%%%%%%%%%%%%%%%%%%%%%%%%%%%%%%%%%%%%%%%%%%%%%%%%%%%%%%%
\section{Video 1}
\subsection{Factores psicológicos}
\begin{enumerate}
    \item Consumo
    \item Negocio 
    \item Precaución
    \item Especulativo
\end{enumerate}
\hlinefill

\begin{itemize}
    \item Habrá inversión hasta donde la eficiencia marginal del capital == taza de interés.
    \item El error del pesimismo: capital tiene mayor rendimiento del que se está especulando tener. 
    \item La teoría asume que el capital es homogéneo pero que difieren en maduración. 
    \item El ahorro tiene un periodo de maduración indeterminada. 
    \item Psicologismo: tendencia subjetiva adbitraria que se torna a convertir en un movimiento de masas por medio de procesos de identificación inconciente, es lo que pasa cuando hay pesimismo.
\end{itemize}

\section{Video 2}
\subsection{Observatorio Market Trends}
\begin{itemize}
    \item Enfoque a variables qualitativas.
    \item Enfocado a dar seguimiento a el ciclo económico mundial, especializado en GT y españa. 
\end{itemize}

\section{Video 3}
Lo mismo que el video 2 solo que en inglés.

\section{Video 4}
\begin{itemize}
    \item Rendimiento marginal del capital
    \item Estado de las espectativas a largo plazo
    \item Teoría general del interés. Mezclado con el tema de la preferencia de la liquidez.
    \item Vinculando la taza de interes es la inversa del rendimiento marginal del capital. 
\end{itemize}


%%%%%%%%%%%%%%%%%%%%%%%%%%%%%%%%%%%%%%%%%%%%%%%%%%%%%%%%%%%%%%%%%%%%%%%%%%%%%%%%%%%%%%%%%%%%%%%%%%%%%%%%%%%%%%%%%%%%%%%%%%%%%%%%%%%%%%%%%%%%%%
\end{document}

