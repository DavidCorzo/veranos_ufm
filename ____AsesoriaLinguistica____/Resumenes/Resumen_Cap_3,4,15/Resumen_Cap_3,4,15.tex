\documentclass[openany]{book}

\usepackage{generalsnips}
\usepackage{calculussnips}
\usepackage[margin = 1in]{geometry}
\usepackage{pdfpages}
\usepackage[spanish]{babel}
\usepackage{amsmath}
\usepackage{amsthm}
\usepackage[utf8]{inputenc}
\usepackage{titlesec}
\usepackage{xpatch}
\usepackage{fancyhdr}
\usepackage{tikz}
\usepackage{hyperref}
\title{Resumen Cap 3,4,15}
\date{2020 June 09, 11:06PM}
\author{David Gabriel Corzo Mcmath}

\begin{document}
\maketitle
%%%%%%%%%%%%%%%%%%%%%%%%%%%%%%%%%%%%%%%%%%%%%%%%%%%%%%%%%%%%%%%%%%%%%%%%%%%%%%%%%%%%%%%%%%%%%%%%%%%%%%%%%%%%%%%%%%%%%%%%%%%%%%%%%%%%%%%%%%%%%%
\chapter{3 Accionar de las máquinas}
\begin{itemize}
    \item El proceso de la escritura es deficil de accionar. 
\end{itemize}
\section{Explorar las circunstancias}
\begin{itemize}
    \item Una situación nos lleva a escribir. 
    \item Para hacer recordatorios, solicitudes, tomar notas, etc. 
    \item Una queja seria puede implicar cambio. 
    \item ``Un problema bien planteado ya esta medio resuelto''
    \item Cuanto más concreta sea la reflexión, más fácil será ponerse a escribir y conseguir un texto eficaz y adecuado a la situación. Demasiadas veces escribimos con una imagen desenfocada del problema, pobre o vaga, que nos hace perder tiempo y puede generar escritos inapropiados e incluso incongruentes
\end{itemize}
\section{Otras maneras de ponerse en marcha}
\begin{itemize}
    \item El bloqueo en la escritura a menudo pasa por que nosotros no escribimos. 
\end{itemize}
\begin{enumerate}
    \item Desarrollar un enunciado: En estas ocasiones hay que basar la reflexión sobre el enunciado. Se trata de desarrollar o expandir las palabras de la pregunta para definirla de manera precisa. 
    \item Diario personal: 10 minutos al día, esto permite accionar la escritura. 
    \item Mapas y redes: forma visual de representar nuestro pensamiento. 
\end{enumerate}


\chapter{Crecimiento de las ideas}
\begin{itemize}
    \item La escritura no solo es guardar información.
    \item Plazma ideas y permite desarrollarlas. 
\end{itemize}
\section{El torbellio de ideas}
\begin{itemize}
    \item Brainstorming 
    \item Pocos segundos o minutos 
\end{itemize}
\subsection{Consejos de torbelio de ideas}
\begin{itemize}
    \item Apuntalo todo 
    \item Creatividad 
    \item No escribir oraciones largas solo es abreviaciones
    \item No importa gramatica u ortografía
    \item Agrupar ideas 
\end{itemize}
\section{Explorar el tema}
\begin{itemize}
    \item La estrella: quien, cuando, donde, que, cuantos, por que.
    \item El cubo: 
\end{itemize}





%%%%%%%%%%%%%%%%%%%%%%%%%%%%%%%%%%%%%%%%%%%%%%%%%%%%%%%%%%%%%%%%%%%%%%%%%%%%%%%%%%%%%%%%%%%%%%%%%%%%%%%%%%%%%%%%%%%%%%%%%%%%%%%%%%%%%%%%%%%%%%
\end{document}

